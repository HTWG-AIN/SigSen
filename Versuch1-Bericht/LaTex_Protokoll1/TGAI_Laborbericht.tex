%---------------
%╔═╗╔═╗╔╦╗╦ ╦╔═╗
%╚═╗║╣  ║ ║ ║╠═╝
%╚═╝╚═╝ ╩ ╚═╝╩  
%---------------
\documentclass[12pt,oneside,a4paper]{report}

% DOCUMENT SETUP
\usepackage[left=3cm, 
			right=2.5cm, 
			top=2.5cm, 
			bottom=2.5cm, 
			includehead, 
			includefoot]{geometry}

% line spacing
\usepackage{setspace}
\setstretch{1,25} % 15/12 --> 1.25

%de­fines Adobe Times Ro­man as de­fault text font
\usepackage{mathptmx}
\usepackage{times} % needed for acronym package

%PDF linking package
\usepackage[hidelinks]{hyperref}

% Language Setup
\usepackage[ngerman]{babel}
% language specific bibliography style
\usepackage[numbers]{natbib}
\usepackage[fixlanguage]{babelbib}
\selectbiblanguage{german}
% bliographystyle setup
% default style names: apalike alphadin ieeetr IEEEtranSN apalike2 alphadin 
% babel specific: babplain, babplai3, babalpha, babunsrt, bababbrv, bababbr3 unsrt 
\bibliographystyle{unsrturl}

% encoding setup
% T1 font encoding for languages that use a latin alphabet
\usepackage[T1]{fontenc} 

% enhanced input encoding handling - utf8 for äÄüÜöÖß...
\usepackage[utf8]{inputenc}
%\usepackage{ucs}%utf8x suppart

% after babel - set chapter string
\AtBeginDocument{\renewcommand{\chaptername}{}}

% enumeration
\usepackage{enumitem}
% tabular extension tabularx
\usepackage{tabularx}

% math packages
\usepackage{amsmath}
\usepackage{nicefrac}
\usepackage{amsthm}
\usepackage{amsbsy}
\usepackage{amssymb}
\usepackage{amsfonts}
\usepackage{MnSymbol}

% patches for latex
\usepackage{fixltx2e}

%special characters
\usepackage{amssymb}
\usepackage{upgreek,textgreek}

% acronym package
\usepackage[printonlyused, footnote]{acronym}

% breakable text in \seqsplit{}
\usepackage{seqsplit}

% \textmu
\usepackage{textcomp}

% package provides a way to compile sections of a document using the same preamble as the main document
\usepackage{subfiles}

% driver-independent color extension - used by listings,tabularx
\usepackage[usenames,dvipsnames,table,xcdraw]{xcolor}

% -- SYNTAX HIGHLIGHTING --
\usepackage{listings}
%\input{cfgs/listings/listings_def_lang_bash-cmd.tex} % adds style BASH_CMD
%\input{cfgs/listings/listings_def_lang_bash-script.tex} % adds style BASH_SCRIPT
\input{cfgs/listings/listings_def_lang_latex.tex} % adds style LATEX
%\input{cfgs/listings/listings_def_lang_matlab.tex} % adds style MATLAB
\input{cfgs/listings/listings_def_lang_python.tex} % adds style PYTHON
%\input{cfgs/listings/listings_def_lang_c++.tex} % adds style CPP
%\input{cfgs/listings/listings_def_lang_c.tex} % adds style C
%\input{cfgs/listings/listings_def_lang_json.tex} % adds style JSON

% HEADLINE CFG
\usepackage{fancyhdr} % Headers and footers
\usepackage{lastpage}
\usepackage{nopageno}
\setlength{\headheight}{1.5cm}
\pagestyle{fancy} % All pages have headers and footers
\fancyhead{} % Blank out the default header
\fancyfoot{} % Blank out the default footer
\fancyhead[L]{}
\fancyhead[C]{}
\fancyhead[R]{}
\fancyfoot[L]{}
\fancyfoot[C]{\thepage}
\fancyfoot[R]{}
% override plain page style for \part, \chapter or 
% \maketitle, which implicit specifies plain page style
\input{cfgs/fancyhdr/fancyhdr_pagestyle_plain.tex}
% set list pagestyle
\input{cfgs/fancyhdr/fancyhdr_pagestyle_lists.tex}

\renewcommand{\chaptermark}[1]{\markright{#1}{}}
\renewcommand{\sectionmark}[1]{\markright{#1}{}}
\renewcommand{\headrulewidth}{0pt}
\renewcommand{\footrulewidth}{0pt}

	
\usepackage{verbatim}
\usepackage{graphicx}
\usepackage{epstopdf}

% floating prevention packages
\usepackage{float}    % used with [H] positioning parameter
\usepackage{placeins} % \FloatBarrier 

% tikz packages
\usepackage{tikz}
\usepackage{caption}
\usepackage[list=true,listformat=simple]{subcaption}
\usepackage{standalone}
\usepackage{pgfplots}


% include only specified tex files - uncommend unneeded
\includeonly{preface/cover,
             preface/abstract,
             preface/tableofcontents,
             preface/listoffigures,
             preface/listoftables,
             preface/lstlistoflistings,
             appendix/bibliography}

%-------------------
%╔═╗╔╦╗╦═╗╦╔╗╔╔═╗╔═╗
%╚═╗ ║ ╠╦╝║║║║║ ╦╚═╗
%╚═╝ ╩ ╩╚═╩╝╚╝╚═╝╚═╝
%-------------------
\newcommand{\strLecture}{Signale, Systeme und Sensoren}
\newcommand{\strDate}{\today}
\newcommand{\strAuthorA}{J. Altmeyer}
\newcommand{\strAuthorB}{M. Kieser}
\newcommand{\strAuthorAEmail}{jualtmey@htwg-konstanz.de}
\newcommand{\strAuthorBEmail}{makieser@htwg-konstanz.de}
% Versuchsbeschreibung 
\newcommand{\strTopic}{Labor Signale, Systeme und Sensoren WS 2015/16}
\newcommand{\strAbstract}{TODO:Zusammenfassung etwa 100 Worte.}
% hyperref customization
\hypersetup{
	pdftitle    ={\strTopic}, % title
	pdfsubject	={\strLecture}, % subject of the document
	pdfauthor	={\strAuthorA, \strAuthorB}, % author
	pdfkeywords	={}, % list of keywords
	pdfcreator	={}, % creator of the document
	pdfproducer	={}, % producer of the document
	colorlinks=false, % false: boxed links; true: colored links
	linkcolor=red, % color of internal links (change box color with linkbordercolor)
    citecolor=green, % color of links to bibliography
    filecolor=magenta, % color of file links
    urlcolor=cyan, % color of external links
	%bookmarks=true, % show bookmarks bar?
	unicode=true, % non-Latin characters in Acrobat’s bookmarks
	pdftoolbar=true, % show Acrobat’s toolbar?
	pdfmenubar=true, % show Acrobat’s menu?
    pdffitwindow=false, % window fit to page when opened
	pdfnewwindow=true % links in new PDF window
}

%-----------------------------------------
% ╔╗ ╔═╗╔═╗╦╔╗╔  ╔╦╗╔═╗╔═╗╦ ╦╔╦╗╔═╗╔╗╔╔╦╗ 
% ╠╩╗║╣ ║ ╦║║║║   ║║║ ║║  ║ ║║║║║╣ ║║║ ║  
% ╚═╝╚═╝╚═╝╩╝╚╝  ═╩╝╚═╝╚═╝╚═╝╩ ╩╚═╝╝╚╝ ╩  
%-----------------------------------------

\begin{document}
\pagenumbering{Roman} 

%\setcounter{section}{0}
\include{preface/cover}

\include{preface/abstract}
\clearpage

%
% TABLE OF CONTENTS
%
\include{preface/tableofcontents}

%
% Abbildungsverzeichnis
%
\include{preface/listoffigures}

%
% Tabellenverzeichnis
%
\include{preface/listoftables}

%
% Listingverzeichnis
%
\include{preface/lstlistoflistings}


%--------------------------
% ╔═╗╦ ╦╔═╗╔═╗╔╦╗╔═╗╦═╗╔═╗ 
% ║  ╠═╣╠═╣╠═╝ ║ ║╣ ╠╦╝╚═╗ 
% ╚═╝╩ ╩╩ ╩╩   ╩ ╚═╝╩╚═╚═╝ 
%--------------------------

\pagenumbering{arabic} 
\setcounter{page}{1}
%
% CHAPTER Einleitung
%
\chapter{Einleitung}
\label{chap:EINL}
\cite{Franz2015j}

In diesem Versuch werden die in der Vorlesung behandelten Techniken zur Kalibrierung, Fehleranalyse und Fehlerrechnung auf den Fall eines Entfernungmessers angewandt. Der Entfernungsmesser basiert auf dem häufig in der Robotik eingesetzten Distanzsensor GP2Y0A21YK0F der Firma Sharp (s. Datenblatt in Moodle), der nach dem Triangulationsprinzip arbeitet.


%
% CHAPTER Versuch 1
%
\chapter{Versuch 1 - Ermittlung der Kennlinie des Abstandssensors}
\label{chap:VERSUCH_1}

\section{Fragestellung, Messprinzip, Aufbau, Messmittel}
\label{chap:VERSUCH_1_FRAGESTELLUNG}

Wie sieht das Verhältnis von Distanz zu Spannung aus dargestellt durch eine Kennlinie? Im Folgenden soll diese Kennlinie mittels
  Messung der Ausgangsspannung des Sensors für 20 verschiedene   Entfernungswerte im Bereich von 10 - 70 cm ermittelt werden. Gemessen werden Distanz, Mittelwert der Spannung und $\Delta$V des Rauschens. Es werden zwei Arten von Messungen durchgeführt eine ohne Berücksichtigung und eine mit Berücksichtigung des Einschwingvorgang des Sensors. 

\paragraph{} Messprinzip: Mit Hilfe des Triangulationsprinzip wird die Distanz über den Sharp Sensor als Spannung in Volt ermittelt. Die Distanz wird mit einem Meterstab gemessen. 
Das DeltaV wird durch die Differenz des Maximalen und Minimalen Spannungswerts ermittelt.

\paragraph{} Aufbau und Messmittel: Der Aufbau der Messeinrichtung ist auf Abbildung \ref{fig:VersuchsAufbauAufg1} zu erkennen. Als Normal wird ein Meterstab verwendet. Ein Brett zur Reflexion des Lichtstrahls des Distanzsensors sowie ein Oszilloskop und Netzgerät.

\begin{figure}[H]
	\centering\small
	\includegraphics[width=\textwidth]{src/VersuchsAufbauAufg1.png}
	\caption{Aufbau des  Versuchs zur Ermittlung der Kennlinie des Abstandssensors}
	\label{fig:VersuchsAufbauAufg1}
\end{figure}


\section{Messwerte}
\label{chap:VERSUCH_1_MESSWERTE}

In Tabelle \ref{tab:Aufg1Tabelle} sind die eigens abgelesenen Werte sowie die des Oszilloskops abgebildet. Letztere bestehen aus dem Mittelwert von 1500 Spannungswerten und ignorieren dabei den Einschwingvorgang des Sensors.

\begin{table}[H]
\begin{tabular}{|r||r|r|r|r|}
\hline 
 & \multicolumn{2}{r|}{abgelesene Werte} & \multicolumn{2}{r|}{Oszilloskop Werte}\\ 
\hline 
Distanz [cm] & Mittelwert [V] & Delta [mV] & Mittelwert [V] & Delta [mV]\\ 
\hline 
10,0 & 1,5 & 80 & 1,48 & 80 \\ 
\hline 
13,2 & 1,3 & 120 & 1,28 & 80 \\ 
\hline 
16,3 & 1,19 & 100 & 1,17 & 80 \\ 
\hline
19,4 & 1,05 & 32 & 1,03 & 32 \\
\hline
22,6 & 0,949 & 32 & 0,93 & 40 \\
\hline
25,8 & 0,881 & 32 & 0,86 & 24\\
\hline
28,9 & 0,82 & 32 & 0,80 &  32\\
\hline
32,1 & 0,778 & 32 & 0,76 &  32\\
\hline
35,3 & 0,714 & 24 & 0,70 & 32 \\
\hline
38,4 & 0,681 & 64 & 0,67 &  64\\
\hline
41,6 & 0,639 & 88 & 0,63 &  64\\
\hline
44,7  & 0,638 & 80 & 0,63 &  64\\
\hline
47,9  & 0,597 & 80 & 0,59 &  72\\  
\hline
51,1 & 0,581 & 40 & 0,57 &  64\\ 
\hline
54,2 & 0,562 & 56 & 0,55 &  56\\ 
\hline
57,4 & 0,539 & 40 & 0,53 &  56\\ 
\hline
60,5 & 0,503 & 56 & 0,50 &  24\\ 
\hline
63,7 & 0,486 & 40 & 0,48 &  24\\ 
\hline
66,8 & 0,469 & 40 & 0,46 &  32\\ 
\hline
70,0 & 0,453 & 40 & 0,44 &  24\\ 
\hline
\end{tabular} 
\caption{Ergebnisse der abgelesenen Werte und Werte des Oszilloskops }
\label{tab:Aufg1Tabelle}
\end{table}

\section{Auswertung}
\label{chap:VERSUCH_1_AUSWERTUNG}
Im Diagramm (Abbildung \ref{fig:Diagramm1}) ist die Osziloskop Kennlinie durchgehend unterhalb der Linie der abgelesenen Daten.

\begin{figure}[H]
	\centering\small
	\includegraphics[width=\textwidth]{src/Diagramm1.png}
	\caption{verlauf}
	\label{fig:Diagramm1}
\end{figure}	

\section{Interpretation}
\label{chap:VERSUCH_1_INTERPRETATION}
Die Differenz zwischen Oszilloskop Werten und abgelesenen Werten lässt sich durch den nicht berücksichtigten Einschwingvorgang der ersten 1000 Daten erklären.


%
% CHAPTER Versuch 2
%
\chapter{Versuch 2}
\label{chap:VERSUCH_2}


\section{Fragestellung, Messprinzip, Aufbau, Messmittel}
\label{chap:VERSUCH_2_FRAGESTELLUNG}
mit ergebnissen aus versuch1 wird gerechnet

\section{Messwerte}
\label{chap:VERSUCH_2_MESSWERTE}
1.Eingangs und Ausgangswerte logarithmiert
  lineare Regression
  Bild Zusammenhang
  
\section{Auswertung}
\label{chap:VERSUCH_2_AUSWERTUNG}
Ermittlung der Kennlinie:

2. logarithmierte Betrachtung y = a+x+b Bild Gerade
	es müssen keine werte entfernt werden. Messwerte bilden eine Gerade. Keine  	    extremen Ausreiser auch nicht bei geringen Spannungen bzw. weiter Entfernung
	y =eb..
3. Steigung a = -1,6 Schnittpunkt mit Y-Achse b = 3
  daraus ergibt sich eine Kennlinie wie folgt: y = ...

\section{Interpretation}
\label{chap:VERSUCH_2_INTERPRETATION}
interpretation der kennlinie: 
aus negativen wert des a ergibt sich abnehmender Zusammenhang der Potenzfunktion : Mit zunehmender Spannung nimmt die Distanz ab




%
% CHAPTER Versuch 3
%
\chapter{Versuch 3}
\label{chap:VERSUCH_3}
Ermittlung des Fehlers
\section{Fragestellung, Messprinzip, Aufbau, Messmittel}
\label{chap:VERSUCH_3_FRAGESTELLUNG}

\section{Messwerte}
\label{chap:VERSUCH_3_MESSWERTE}
Messung des DinA4 Blattes in tabelle2

\section{Auswertung}
\label{chap:VERSUCH_3_AUSWERTUNG}

\section{Interpretation}
\label{chap:VERSUCH_3_INTERPRETATION}
1cm Systematischer Fehler


%
% CHAPTER Versuch 4
%
\chapter{Versuch 4}
\label{chap:VERSUCH_4}

\section{Fragestellung, Messprinzip, Aufbau, Messmittel}
\label{chap:VERSUCH_4_FRAGESTELLUNG}

\section{Messwerte}
\label{chap:VERSUCH_4_MESSWERTE}

\section{Auswertung}
\label{chap:VERSUCH_4_AUSWERTUNG}

\section{Interpretation}
\label{chap:VERSUCH_4_INTERPRETATION}
%
% CHAPTER Anhang
%
\renewcommand\thesection{A.\arabic{section}}
\renewcommand\thesubsection{\thesection.\arabic{subsection}}

\chapter*{Anhang}
\label{chap:APPENDIX}
\addcontentsline{toc}{chapter}{Anhang}
%\setcounter{chapter}{0}
\addtocounter{chapter}{1}
\setcounter{section}{0}

\section{Quellcode}
\label{chap:APPENDIX_SOURCECODE}

\subsection{Quellcode Versuch 1}
\label{chap:APPENDIX_SOURCECODE_V1}

\subsection{Quellcode Versuch 2}
\label{chap:APPENDIX_SOURCECODE_V2}

\subsection{Quellcode Versuch 3}
\label{chap:APPENDIX_SOURCECODE_V3}

\subsection{Quellcode Versuch 4}
\label{chap:APPENDIX_SOURCECODE_V4}


\section{Messergebnisse}
\label{chap:APPENDIX_MEASUREMENT_SOURCE}

%
% Literaturverzeichnis
%
\include{appendix/bibliography}

\end{document}
%------------------------------------
% ╔═╗╔╗╔╔╦╗  ╔╦╗╔═╗╔═╗╦ ╦╔╦╗╔═╗╔╗╔╔╦╗
% ║╣ ║║║ ║║   ║║║ ║║  ║ ║║║║║╣ ║║║ ║ 
% ╚═╝╝╚╝═╩╝  ═╩╝╚═╝╚═╝╚═╝╩ ╩╚═╝╝╚╝ ╩ 
%------------------------------------